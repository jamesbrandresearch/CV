%% J Brand CV template


% Copyright (C) 2004-2009 Jason Blevins <jrblevin@sdf.lonestar.org>
% http://jblevins.org/projects/cv-template/
%
% You may use use this document as a template to create your own CV
% and you may redistribute the source code freely. No attribution is
% required in any resulting documents. I do ask that you please leave
% this notice and the above URL in the source code if you choose to
% redistribute this file.

\documentclass[letterpaper]{article}

\usepackage{hyperref}
\usepackage{geometry}
\usepackage{textcomp}
\usepackage{setspace}

% Comment the following lines to use the default Computer Modern font
% instead of the Palatino font provided by the mathpazo package.
% Remove the 'osf' bit if you don't like the old style figures.
\usepackage[T1]{fontenc}
\usepackage[sc,osf]{mathpazo}

\def\name{James Brand}

% Replace this with a link to your CV if you like, or set it empty
% (as in \def\footerlink{}) to remove the link in the footer:
\def\footerlink{}

% The following metadata will show up in the PDF properties
\hypersetup{
  colorlinks = true,
  urlcolor = black,
  pdfauthor = {\name},
  pdfkeywords = {psychology, language, development},
  pdftitle = {\name: Curriculum Vitae},
  pdfsubject = {Curriculum Vitae},
  pdfpagemode = UseNone
}

\geometry{
  body={6.5in, 8.5in},
  left=1.0in,
  top=1.25in
}

% Customize page headers
\pagestyle{myheadings}
\markright{\name}
\thispagestyle{empty}

% Custom section fonts
\usepackage{sectsty}
\sectionfont{\rmfamily\mdseries\Large}
\subsectionfont{\rmfamily\mdseries\itshape\large}

% Other possible font commands include:
% \ttfamily for teletype,
% \sffamily for sans serif,
% \bfseries for bold,
% \scshape for small caps,
% \normalsize, \large, \Large, \LARGE sizes.

% Don't indent paragraphs.
\setlength\parindent{0em}

% Make lists without bullets
\renewenvironment{itemize}{
  \begin{list}{}{
    \setlength{\leftmargin}{1.5em}
  }
}{
  \end{list}
}

\begin{document}

%%%%% Header
% Place name at left
%{\huge \name}
% Alternatively, print name centered and bold:
\centerline{\huge \bf \name}
\vspace{0.25in}


 \normalsize
 Fylde College\\
 Department of Psychology\\
  \href{http://www.lancaster.ac.uk}{Lancaster University}\\
  Lancaster\\
  LA1 4YF\\
  
  \begin{minipage}{0.45\linewidth}
%\large
Email: \href{mailto:j.brand@lancaster.ac.uk}{\tt j.brand@lancaster.ac.uk}\\
Homepage: \href{https://jamesbrandresearch.wordpress.com}{\tt https://jamesbrandresearch.wordpress.com}\\
 
\end{minipage}


%%%%% Education
\section*{Education}

\begin{itemize}
  \item Ph.D. (in progress), Psychology, Lancaster University, expected Autumn 2016
  
  \item M.Sc., Evolution of Language and Cognition, University of Edinburgh, November 2012
  
  \item B.A., English Language, Sheffield Hallam University, June 2010
  
  \item Level 5 Certificate in Teaching English to Speakers of Other Languages, Sheffield Hallam University, March 2010

  

\end{itemize}

%%%%% Research interests
\section*{Research Interests}

How do cognitive biases shape our language? My research focuses on the different ways that our language is shaped by the brain, addressing how language acquisition and use can explain many fundamental properties of language. I adopt a variety of experimental techniques (such as cross-situational, artificial language and iterated learning paradigms) which is complimented by corpus based analyses, in order to produce robust and comprehensive empirical evidence. This approach is highly multidisciplinary, with the aim being to answer and generate research questions that have implications for areas of psychology, linguistics and the cognitive sciences.

%%%%% Academic Honors
\section*{ Honors and Awards}
\begin{itemize}
\item Experimental Psychology Society Grindley Grant (2015)
\item Lancaster University Depermental studentship (2013-2016)
\end{itemize}

%%%%% Papers
\section*{Publications}

\onehalfspacing

  \hangindent=.7cm {\bf Brand, J.,} Monaghan, P., \& Walker, P. (Under review). The changing role of form-to-meaning mappings for the growing vocabulary.

   
 \singlespacing
 
 %%%%% Talks and Presentations
\section*{Talks}
\onehalfspacing

\hangindent=.7cm {\bf Brand, J.,} \& Monaghan, P. (2016, May). Predictors of lexical stability in an artificially learnt language. Talk presented at Psycholinguistics in Flanders, Antwerp, Belgium.

\hangindent=.7cm {\bf Brand, J.,} Monaghan, P., \& Walker, P. (2015, June). The changing role of form-to-meaning mappings for the growing vocabulary. Invited talk to the Phonetics Reading Group. Lancaster University, UK.

\hangindent=.7cm {\bf Brand, J.,} Monaghan, P., \& Walker, P. (2015, June). The changing influence of form-meaning mappings during vocabulary development. Invited talk at the LangNet Workshop on Second Language Acquisition and First Language Acquisiton. Lancaster University, UK.

\hangindent=.7cm {\bf Brand, J.,} Monaghan, P., \& Walker, P. (2014, July). Iconicity and arbitrariness in language learning: Does vocabulary size matter? Talk presented at the 5th UK Cognitive Linguistics Conference. Lancaster University, UK.

\hangindent=.7cm {\bf Brand, J.,} Monaghan, P., \& Walker, P. (2014, May). The changing influence of form-meaning mappings during vocabulary development. Talk presented at Alston Hall Child Language Workshop. Alston Hall, UK.

 \singlespacing
\section*{Other Presentations}

\onehalfspacing

\hangindent=.7cm {\bf Brand, J.,} \& Monaghan, P. (2016, August). Predictors of lexical stability in an artificially learnt language. Poster presented at the Annual Meeting of the Cognitive Science Society, Philadelphia.

\hangindent=.7cm {\bf Brand, J.,} Monaghan, P., \& Walker, P. (2015, October). The changing influence of form-meaning mappings during vocabulary development. Poster presented at 1st LuCiD Mini-conference. Liverpool, UK.

\hangindent=.7cm {\bf Brand, J.,} Monaghan, P., \& Walker, P. (2015, January). The changing influence of form-meaning mappings during vocabulary development. Poster presented at Experimental Psychology Society London Meeting. London, UK.

\hangindent=.7cm {\bf Brand, J.,} Monaghan, P., \& Walker, P. (2013, December). Iconicity and arbitrariness in language learning: Does vocabulary size matter? Poster presented at Faculty of Science and Technology Christmas Conference. Lancaster University, UK.


 
 
 \singlespacing

%%%%% Professional Activities and Service
\section*{Professional Activities and Service}
\begin{itemize}
\item Organiser at the 5th Implicit Learning Seminar, Lancaster University, June 2016
\item Set up the Lancater R User Group (LancsR) and Lancater Psychology GitHub, 2016
\item ESRC International Centre for Language and Communciative Development Knowledge Exchange and Training Commitee, 2014 - 2016
\item  Organiser of the Lancaster University Psychology Department Internal Seminar Series, 2014 - 2015
\end{itemize}

%%%%% Public Engagement and Outreach
\section*{Public Engagement and Outreach}
\begin{itemize}
\item Presenter at LuCiD and Lancaster BabyLab Campus in the City, May 2016
\item Presenter at 'Kids say the funniest things' at the ESRC Festival of Social Sciences. Manchester Museum, UK, November 2015
\item Supervisor for South Lakes Federation Poster competition, July 2015
\item Presenter at LuCiD and Lancaster BabyLab Campus in the City, April 2015
\item Presenter at Lancaster University's UCAS (Universities and Colleges Admissions Sevice) open day, 2013 - 2014
\end{itemize}

%%%%% Teaching
\section*{Teaching}
\begin{itemize}
\item Teaching Assistant, 2nd Year Undergradaute Statistics (Instructor: Dr. Michelle To), 2015 - 2016
\item Teaching Assistant, 1st Year Undergradaute Statistics (Instructor: Dr. Callum Hartley), 2013 - 2016
\item Tutor, 2nd Year Undergradaute Statistics Surgery, 2016
\item Teaching Assistant, M.Sc. Analysing and Interpreting Psychological Data (Instructors: Dr. Michelle To and Dr. Robert Davies), 2015
\item Teaching Assistant, 1st Year Undergradaute Psychological Approaches, 2013 - 2015
\item Guest Lecturer, 1st Year Undergraduate Revision Session, 2014
\item Seminar Tutor, 2nd Year Undergradaute Cognitive Psychology, 2014
\item Seminar Tutor, 1st Year Undergradaute Understanding Psychology, 2013 - 2014
\item Tutor, 1st Year Undergradaute Statistics Surgery, 2014


\end{itemize}

%%%%%% Computing Proficiency
\section*{Computing Proficiency}
\begin{itemize}
\item Proficient in using R (packages such as ggplot2, lme4, stringdist, effects, plyr)
\item Proficient in PsychoPy
\item Proficient in E-prime
\item Proficient in LaTeX
\item Proficient in SPSS
\item Proficient in Microsoft Office applications (Excel, PowerPoint, Word)
\item Familiar with Praat
\item Familiar with Python
\end{itemize}

\section*{Afilliations}
\begin{itemize}
\item Present:
\item ESRC International Centre for Language and Communicative Development (LuCiD)
\item Language and Communication Lab, Lancaster University
\item Member of the Cognitive Science Society
\\
\item Past:
\item Centre for Research in Human Devlopment and Learning, Lancaster University
\item Language Evolution and Computaion Research Unit, University of Edinburgh
\item European Human Behaviour and Evolution Association
\end{itemize}


\bigskip

% Footer
\begin{center}
  \begin{footnotesize}
    Last updated: \today \\
    \href{\footerlink}{\texttt{\footerlink}}
  \end{footnotesize}
\end{center}

\end{document}